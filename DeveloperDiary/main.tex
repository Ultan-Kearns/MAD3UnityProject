\documentclass{report}
\usepackage[utf8]{inputenc}
\usepackage{graphicx}
\usepackage{listings}
\usepackage{hyperref}
\usepackage{amsmath}
\title{Developer Diary}
\author{Ultan Kearns}
\date{September 2019}
\begin{document}
\maketitle
\tableofcontents
\input{chapters.tex}
\section{What is this game?}
The game is a 2D platformer called Quickfire.  
The object of the game is 
to survive as many rounds as possible while bullets come flying at the player, the bullets will get faster with each level. At the beginning of the game the player will have a set number of lives and the object of the game is to avoid dying and to get a high score.
\subsection{What is a platformer?}
A platformer is a game in which the objective of the game is to jump on platforms and reach the end of the level.  There are usually many obstacles which the player must face in order to advance through the level usually these are gaps, spikes or any other form of object which will either kill the player or damage his health.
\subsection{Not a typical platformer}
This game deviates from the standard platformer in a very original way as it has bullets which the player must avoid in order to progress through the level.  If the player is hit by a bullet or the player falls off the platform the game end and the players score is saved as the highscore if they have a score greater than the previous high score.
\subsection{Aim of game}
The aim of the game is to survive wave after wave of bullets and accumulate as many points as possible.  The game also will get progressively more difficult as the player advances in order to give more of a challenge to skilled players.
\section{About this game}
 This game was developed using the unity game engine.  The game engine helped me a lot while developing this game and made programming the physics of the game so much easier.  I started out with an empty scene and quickly I found this game took on a life of it's own.  I hope you enjoyed playing it as much as I did developing it.  My aim when making this game was to make an intuitive and challenging 2D platformer and to implement all the requirements as specified in the design document for this game.  I found the design document to be very helpful and it really was comprehensive and included all the information I needed to make a great 2D platformer.
 \subsection{Key points to know about this game}
 \begin{itemize}
     \item It's a 2D platformer
     \item The object of the game is to accumulate as many points as possible by surviving for a prolonged period of time and to achieve the high score
     \item This game was developed using the unity engine
 \end{itemize}
 \section{Experience with designer}
 The designer of the game was very helpful with all of my questions which were limited due to a very comprehensive design document and test plan.  He was also very quick to design the game assets and send them to me promptly.  I found working with the designer a great experience which will add to my skills as a developer, as dealing with clients is one of the key skills a developer can have in the modern world of software development.
\chapter{Planning}
\subsection{Reading the design document}
Before I started coding I read the design document extensively and made a note of all the requirements which the game must meet.  I met with the designer to discuss the key requirements of the game and discussed how they would like me to implement them.
\section{Beginning the project}
I started this project off with brainstorming ideas of how to implement the various features that the designer had specified in the design document. I thought of how best to implement them so that the movement and gameplay would seem fluid and the controls intuitive. I also brainstormed what type of features could be improved upon and discussed these with the designer. I read the design document a few times to get the general feel for the game and to ensure I understood all the requirements for the game.  I began by making a simple prototype and slowly improving upon it gradually.  I discussed any changes I wished to make with the designer and ensured that they agreed with the changes before I began implementing them into the game.
\chapter{Implementation}
\section{Beginning implementation}

\subsection{Starting to code}
When I started to code the project I had many issues such as boxcolliders not working and various other problems luckily these weren't  hard to solve as the unity documentation site here:\href{https://docs.unity3d.com/Manual/index.html}{Link to Unity documentation} is fairly extensive.  I was able to solve a vast majority of these problems by analyzing code from the labs and was able to find solutions to my specific problems within the game I'm developing. 
\end{document}
